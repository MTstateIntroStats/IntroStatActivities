\def\theTopic{Reading 10}

\section{ Comparative Studies}

  With the textbook cost we combined all MSU students together and did
  not try to compare parameter values across groups.  In many
  situation, however, the point of a study is to compare two or more
  groups. Here are some such example studies for you to consider. 

\begin{enumerate}
  \item  Depression is a serious problem which affects millions of
    people each year. Suppose that you are asked to 
    design a survey to compare answers of men and women to this question:\\
    {\sf If you were feeling depressed, would you visit MSU Counseling Services?}
    \begin{enumerate}
    \item How would you select male and female MSU students to
      interview?
\begin{students}
        \vfill
\end{students}
\begin{key}
 {\it AWV. They should try to get a representative sample.}
\end{key}
\item What are the variables you would collect in each interview?
      Are they categorical or quantitative? Is one explanatory? Is one a response?
\begin{students}
        \vfill
\end{students}
\begin{key}
 {\it Gender (Categorical -- possibly explanatory) and would you use
   the MSU SHS (categorical -- the response).}
\end{key}
    \item The statistical inference tools you learned in Unit 1 do not
      quite apply to this situation.  Why not? 
\begin{students}
        \vfill
\end{students}
\begin{key}
 {\it So far we've only dealt with a single proportion, and have not
   compared two groups. }
\end{key}
    \item Would this be an experiment or an observational study?
\begin{students}
        \vfill
\end{students}
\begin{key}
 {\it Observational study}
\end{key}
    \item What would be your scope of inference?
\begin{students}
        \vfill
\end{students}
\begin{key}
 {\it We will not be able to say that gender causes who would seek
   help at MSU SHS because we did not randomly assign gender.  If you
   randomly selected students from the MSU population, then we can
   extend the inference back to the population. Otherwise, it's just
   valid within the sample.}
\end{key}

    \end{enumerate}

  \item In a clinical trial 183 patients with chronic asthma were
    randomly assigned  to either placebo (n = 92) or budesonide (n = 91). After 12 weeks
    of treatment, doctors measured their lung function (Forced
    Expiration Volume in 1 second, FEV$_1$) in cc's.      
   \begin{enumerate}
    \item How do you think these patients were selected to be in the
      study? Is there a larger population they were drawn from?
\begin{students}
        \vspace*{\fill} \newpage
\end{students}
\begin{key}
 {\it They must have been referred to the study by a doctor, so they
   were not chosen at random.}
\end{key}

    \item What are the variables mentioned?
      Are they categorical or quantitative? Is one explanatory? Is one a response?
\begin{students}
        \vfill
\end{students}
\begin{key}
 {\it FEV is a quantitative response. Treatment (placebo or
   budesonide) is the categorical explanatory variable. }
\end{key}

    \item Would it be appropriate to ``block'' the patients before
      randomly assigning treatments?\begin{students}
        \vfill
\end{students}
\begin{key}
 {\it Yes, it would be good to use a pretest on FEV in order to be
   sure we have similar numbers of poor, medium, and healthy}
\end{key}

    \item What parameters would you compare between treatment and
      control groups?
\begin{students}
        \vfill
\end{students}
\begin{key}
 {\it Compare $\mu_{control}$ to $\mu_{treated}$.}
\end{key}

    \item Was this  an experiment or an observational study? 
\begin{students}
        \vfill
\end{students}
\begin{key}
 {\it a randomized experiment}
\end{key}

    \item What would be the scope of inference?
 \begin{students}
        \vfill
\end{students}
\begin{key}
 {\it We can make causal inference about the effect of the new drug on
 asthma symptoms in the sample.}
\end{key}

    \end{enumerate}
  \item Key Points:  What are two main differences between studies 1
    and 2, and how do they affect the ``Scope of inference'' for each
    study? \vfill
  \item In the last 10 years, the proportion of children who are
    allergic to peanuts has doubled in Western countries. However,
    the allergy is  not very common in some other countries where peanut
    protein is an important part of peoples' diets.   \\
     The LEAP randomized trial, reported by  Du Toit, et.al in the
     {\it  New England Journal of  Medicine} in February 2015
     identified over 500 children ages 4 to 10 months who showed some
     sensitivity to peanut protein. They randomly assigned them to two
     groups:
     \begin{itemize}
     \item Peanut avoiders: parents were told to not give their kids
       any food which contained peanuts, and
     \item Peanut eaters: parents were given a snack containing 
       peanut protein and told to feed it to their child several times
       per week (target dose was at least  6g of peanut protein per week).
     \end{itemize}
      At age 5 years, children were tested with a standard skin prick
      to see if they had an allergic reaction to peanut protein (yes
      or no).
      \begin{enumerate}
      \item What variables were measured? 
      Are they categorical or quantitative? Is one explanatory? Is one
      a response?
\begin{students}
        \vfill
\end{students}
\begin{key}
 {\it Peanut consumer or avoider (categorical, explanatory) and
   Allergic or not (categorical response).}
\end{key}

    \item Was this an experiment? was it randomized? were subjects
      blinded to the treatment?
\begin{students}
        \vspace*{\fill}
\end{students}
\begin{key}
 {\it Randomized experiment. Subject could not be blinded because of
   the diets involved.}
\end{key}

      \end{enumerate}

\end{enumerate}