\def\theTopic{Reading 19}

\section{ Smell of Baking Bread}


\begin{center}
\vspace*{.1in}
{\bf {\large Are we better people in a pleasant environment?}}
\end{center}
\vspace{-.1in}

A study from the {\it Journal of Social Psychology}\footnote{ Nicolas Gu\'eguen, 2012.  The Sweet Smell of \ldots Implicit Helping:
 Effects of Pleasant Ambient Fragrance on Spontaneous Help in Shopping
 Malls .  {\it Journal of Social Psychology} {\bf152}:4, 397-400 }
 reports on a study of
people's behavior under two different conditions. The researchers gave
this description of their methods: 
   \quotation{ \small
``The participants were 200 men and 200 women (between the ages of
approximately 20 and 50) chosen at random while they were walking in a
large shopping mall. The participant was tested while walking near
areas containing pleasant ambient odors (e.g.: bakeries, pastries) or
not (e.g. clothing stores). Four young women (M = 20.3 years) and
four young men (M = 21.3 years) served as confederates in this
study. They were dressed in clothing typically worn by people of this
age (jeans/T-shirt/boat shoes). The confederate chose a participant
walking in his/her direction while standing in front of a store
apparently looking for something in his/her bag. The confederate was
carefully instructed to approach men and women walking alone,
apparently aged from 20 to 50, and to avoid children, adolescent, and
elderly people. The confederate was also instructed to avoid people
who stopped near a store. Once a participant was identified, the
confederate began walking in the same direction as the participant
about three meters ahead. The confederate held a handbag and
accidentally lost a glove. The confederate continued, apparently not
aware of his/her loss. Two observers placed approximately 50 meters
ahead noted the reaction of the passer-by, his/her gender, and
estimated, approximately, his/her age. Responses were recorded if the
subject warned the confederate within 10 seconds after losing the
object. If not, the confederate acted as if he/she was searching for
something in his/her hand-bag, looked around in surprise, and
returned to pick up the object without looking at the
participant.''\footnote{ibid}} 
  
Assume that when the confederate saw a person who fit the qualifications, a
coin was flipped. If it came up Heads, the subject was picked to be in
the study, if Tails, they were skipped.

Data summary:\\
 In the bakery group of 200 subjects, 154 people stopped the ``confederate'' to tell
 them they had dropped something.\\
 Outside the clothing store, 104 of the subjects stopped the
 ``confederate''.  
\begin{center}
  {\large\bf Important Points}
\end{center}


\begin{itemize}
\item  What question did the researchers want to answer?
\begin{students}
 \vspace{.8cm}
\end{students}

\begin{key}
{\it Does the smell of baking bread influence people to be more altruistic?}
\end{key}


\item Who were the subjects?
\begin{students}
 \vspace{.8cm}
\end{students}

\begin{key}
{\it 400 people in a mall}
\end{key}


\item Were they randomly selected?
\begin{students}
 \vspace{.8cm}
\end{students}

\begin{key}
{\it Yes, by coin flip to be in or not.}
\end{key}



\item Were treatments applied at random?
\begin{students}
 \vspace{.8cm}
\end{students}

\begin{key}
 {\it No. subjects were either by the clothing store or by the bakery.}
\end{key}

\item What two measurements do we have for each subject in the study?
  Which variable is the response?  
\begin{students}
 \vspace{.8cm}
\end{students}

\begin{key}
 {\it Bakery/clothing setting and helped/didn't help (the resposne)}
\end{key}

\item Thinking back to Unit 2, which situation (on the review table)
  are we using?
\begin{students}
 \vspace{.8cm}
\end{students}

\begin{key}
 {\it Two proportions.}
\end{key}




\end{itemize}



