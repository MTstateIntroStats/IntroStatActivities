\def\theTopic{Reading 12}

\section{Energy Drinks - Reading}

In the next activity, we will use data  from this study:\\

 Curry K, Stasio MJ.  (2009). The effects of energy drinks alone and with
 alcohol on neuropsychological functioning. {\it Human  Psychopharmacology}.
{\bf 24}(6):473-81. 

Researchers used 27 volunteer college students from a private college
in the US. All were women because the college has 3 times as many
women as men, and the researchers thought it would be too hard (take
too long)  to get enough of both sexes. They advertised for
volunteers, obtained consent to drinking beverages with alcohol and or
caffiene, and were told that the study involved measuring creativity
effects. \\
The abstract says:

\begin{quotation}
  In a double-blind, placebo-controlled design, 27
  non-caffeine-deprived female participants were randomly assigned to
  consume a caffeinated energy drink alone (RED), one containing
  alcohol (RED+A), or a non-alcoholic, non-caffeinated control
  beverage. Pre- and post-test assessments were conducted using
  alternate forms of the Repeatable Battery for the Assessment of
  Neuropsychological Status (RBANS).
\end{quotation}

RBANS comes in two forms (A and B) which have been shown to give the
same scores to people in terms of cognitive status.  One form of the
test was administered several days before the subjects were given the
treatment, and another was given 30 minutes after they had drunk the
assigned beverage.  The response we will look at is the change in
RBANS scores (post score minus pre score). Initially we'll compare
only the control and RED+A groups.

The RBANS test is called a ``battery'' because it is intended to measure
many aspects of human cognition including: Immediate memory,
Visuospatial/constructional, Language, Attention, and  Delayed
memory. The scores we will examine are combined (Total scale) across
all those dimensions.  

Two of the drinks were commercial products: RED = Green Monster
(caffienated) and RED+A = Orange Sparks (6\% alcohol and caffiene)
while the control was colored Diet 7-Up.
\newpage
Questions:
\begin{enumerate}
\item What does ``double-blind'' mean for this study? \vfill
\item What are the variables of interest? What type are they?\vfill
\item Was this an experiment or an observational study?\vfill
\item Were subjects randomly chosen from some larger population of
  people?\vfill
\item If we find strong evidence of an association between the drinks
  and the ``change in RBANS'' response, what will the scope of
  inference be?  (Clearly explain and include your reasoning.)
\end{enumerate}
\normalsize