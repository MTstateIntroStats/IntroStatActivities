\def\theTopic{Reading 9}
%%  Reordering for summer 2016 renumbered 8 as 9

\section{ Experiments and Observational Studies}

\subsection{ More Types of Variables}


Sometimes a change in one variable causes another variable to change. 
For example, \vspace{-.18in}
\begin{itemize}
\item hours spent studying might affect a person's
grade in statistics
\item weight of a car might affect the gallons of fuel needed to go 100 miles
\item major in college might affect ``employment in field'' six months
  after graduation. It might also affect starting salary.
\end{itemize}
In these cases, we have an {\bf explanatory} variable which has an
effect on a  {\bf response} variable.  \\
In other cases, variables are simply {\bf associated}.  For example: \vspace{-.18in}
\begin{itemize}
\item weight and length of Rainbow trout
\item population of a city and the amount of taxes collected
\item number of beds  and  number of patient deaths per year
  in  hospitals
\item cultivated acres and annual agricultural income  in  counties
\end{itemize}
Another variable might be causing changes in both (for example the age
of the trout would affect both weight and length), or the connection might be
more complex.  Throughout this course we will be looking for
associations and trying to determine if one variable is {\bf causing}
changes in the other.  \\
{\bf Caution:} We will often find that variables are associated (or
  not), but {\bf Association does not imply causation!}   You may have
  heard people say: ``Correlation does not imply causation.''  They
  are trying to say the same thing, but in statistics, correlation has
  a very specific meaning. It is the strength (and direction) of a
  linear relationship between two quantitative variables and should
  not be used when one variable is (or both are) categorical.  For now
  just say ``association'' instead. We will get to correlation in a
  few weeks.\\
For the following scenarios, circle any explanatory variable (if there
is one) and draw an arrow to the response variable (if there is
one). % If neither affects the other, put a rectangle around each.
\begin{enumerate}
\item Textbooks:  cost, type of binding, number of pages\\
\item Chickens:  breed, diet,  weight gain\\
\item Test Scores:  Exam 1, Exam 2, Exam 3\\
\end{enumerate}


\subsection{ Types of Studies}

  Because we do often want to say that changing one variable causes
  changes in another, we need to distinguish carefully between two
  types of studies. One will allow us to conclude that there is a causal
  link;   others will not.

  {\bf Experiments} are studies in which treatments are assigned to
  the units of the study. Units (or cases) can be people -- called
  subjects -- or things like animals, plants, plots of ground.  A
  treatment variable has {\bf levels} like different drugs or dosages
  or times or oven temperatures, so it is a categorical variable. \\
  {\bf Randomized Experiments} are those in which treatments are {\bf
    randomly} assigned.  These are very special -- we'll keep an eye
  out for them.


  {\bf Observational Studies} are studies in which no variables are
  set, but each is simply observed, or measured, on the units (or
  cases). The variables could be categorical or quantitative and the
  units, as with experiments, can be subjects (people), animals,
  schools, stores, or any other entity that we can measure. 

  The next activity will discuss the differences between experiments
  and observational studies.  You'll need a bit more information about
  how experiments are set up.

 \subsection { Four Principles of Experimental Design}

  {\bf Control}.  We want the different treatments used to be as
  similar as possible.  If one group is getting a pill, then we give
  a ``control'' group a {\bf placebo} pill which is intended to have no
  physical effect.  The placebo may have a very real psychological effect 
  because often subjects who believe that they are getting a drug 
  do actually improve.  To avoid these complexities, it is also
  important that subjects be ``{\bf blind}'' to treatment -- that they
  not be told which treatment they were assigned, and, in cases where
  measurements are subjective, that the raters also not know who
  received which treatment (a double--blind study is when neither
  subject nor rater knows which treatment was assigned).  
  \\
  In an agricultural experiment which involves spraying a fertilizer
  or an herbicide, the ``control'' plots would also be sprayed, but just
  with water. 

  {\bf Randomization}.  When doctors first started applying treatments,
  they thought that they should choose which treatment was given to each
  patient. We'll soon see that such choices lead to biased results, and
  that randomization is a better strategy because it makes treatment
  and control groups most similar in the long run.  Statisticians
  recommend that we randomize whenever possible, for example, we like
  to randomize the order in which we make measurements.

  {\bf Replication}. Larger sample sizes give more accurate results,
  so we do want to make studies as large as we can afford them to
  be. Additionally, a principle of science is that  whole
  studies should be replicated.  With people, one study most often uses a
  very narrow slice of the human population, so it's a very good idea
  to run it again in a different country.  The Skeptic reading from
  page \pageref{skeptoid} mentioned ``meta analysis'' which combines results from
  multiple ESP experiments to broaden the inferences we can make about
  ESP. 

  {\bf Blocking}. Agronomists comparing yields of different wheat
  varieties (treatments), have to worry about the fact that in any
  field there are patches which are more (or less) productive than
  others. Instead of randomly assigning varieties to plots across a
  large field, they split the field into ``blocks'' which contain more
  uniform plots, and randomly assign varieties within each block. \\
  In a comparison of a new surgical technique with an old one, doctors
  might split patients into three different risk levels and assign
  treatments randomly within each group (block).  Then each treatment
  group has roughly the same number of high, medium, and low risk
  patients, and we will get a comparison of the treatments which is
  not ``confounded'' with risk level.  Studies are often blocked by
  gender or age, as well.\\
  Two variables are ``confounded'' if their effects cannot be
  separated.  For example, the first times we offered this curriculum,
  students taking a Tuesday--Thursday class all had the new curriculum
  in TEAL rooms, and the students taking MWF classes all had the old
  curriculum in non-TEAL rooms.  We saw a large improvement in average
  test scores, but could not separate the effects of ``day of week'',
  curriculum, and room type.


{\bf Important Points}:
\begin{itemize}
\item Know the difference between an explanatory variable and a
  response.\vfill
\item What is the difference between an experiment and an
  observational study?\vfill
\item How could MSU run a randomized experiment to compare two different
  types of STAT 216 curriculum?  How would we  set that up?\vfill
\item Why do clinical trials expect doctors to use a double-blind protocol?\vfill
\item What are two ways in which replication is used? \vfill
\item Suppose we want to compare two weight reduction plans.  How
  would blocking on gender change the set up of the experiment?    \vspace*{\fill}
\end{itemize}
