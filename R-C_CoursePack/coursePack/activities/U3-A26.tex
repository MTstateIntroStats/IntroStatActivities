\def\theTopic{Pairing }
\def\dayNum{26 }


\section{ Are You Two Together?  -- Pairing}


  % When we want to make comparisons, and the data include lots of
  % excess noise, {\bf pairing} is a way to get ``cleaner'' or more
  % precise comparisons.

   In the ``Energy Drinks'' study we've looked at
  several times, each participant took a test  before the
  experiment really began. She  returned about 10 days later, and was
  randomly assigned a treatment (RED, REDA or Control) and was again
  given a very similar test.  The response variable we looked at
  was her ``change in RBANS'', meaning we subtracted the first test
  score from the second.   
  \begin{enumerate}
  \item The ``Repeatable Battery for the Assessment of
    Neuropsychological Status'' outputs scores for immediate memory,
    visuospatial/construction, language, attention, and delayed
    memory.  Discuss: What attributes of a person would make her score
    higher or lower?  Write down two or three of your best guesses.
\begin{students}
    \vspace{2cm}    
\end{students}

\begin{key}
  {\it Intellect, sleepiness, memory, motivation.}
\end{key}


%   \item Write down an estimate of how much a person's attribute will
%     change in 10 days for each of the attributes above.  Think of how
%     much one person changes relative to differences between different
%     people (take $\sigma = 4$ to be the spread from person to person). \begin{students}
%     \vspace{2cm}    
% \end{students}

% \begin{key}
%   {\it Intellect and memory do not change much from day to day
%     motivation could change a fair amount, and  sleepiness could
%     change a lot.}
% \end{key}


The ``Repeatable'' part of RBANS means that it comes in several
different versions which are all supposed to give the same scores to
the same people.  If the researchers had used exactly the same
questions, subjects might have learned from the first attempt and done
better the second time.


  \item  For each of the following situations, write ``paired'' if
    there is one sample measured twice or ``two samples'' otherwise.
    Ask your self: ``Does it make sense to take differences and
    analyze those? (paired) Or do we compute means from each of two groups?''
     Another clue: If sample sizes might be different, it's not
     paired. 
 
    \begin{enumerate}
    \item To study the effect of exercise on brain activity
      researchers recruited sets of identical twins in middle age.
      One twin was randomly assigned to engage in regular
      exercise and the other didn't exercise. 
\begin{students}
 \vspace{1cm}
\end{students}

\begin{key} Paired -- a pair of twins is the ``unit''
\end{key}


    \item To see if curriculum changes are effective, researchers took
      a sample of 100 eighth graders' standardized test scores from
      this year and compared them to a sample of 100 scores of last
      year's eighth graders (on the same exam).
\begin{students}
 \vspace{1cm}
\end{students}

\begin{key} Not paired -- 2 samples.
\end{key}


    \item In a study to determine whether the color red increases how
      attractive men find women, one group of men rated the
      attractiveness of a woman after seeing her picture on a red
      background and another group of men rated the same woman after
      seeing her picture on a white background.  
\begin{students}
 \vspace{1cm}
\end{students}

\begin{key} Not paired. The men form two samples - or treatment groups.
\end{key}


    \item To measure the effectiveness of a new teaching method for
      math in elementary school, each student in a class getting the
      new instructional method is matched with a student in a separate
      class on IQ, family income, math ability level the previous
      year, reading level, and all demographic characteristics.  At
      the end of the year, math ability levels are measured.
\begin{students}
 \vspace{1cm}
\end{students}

\begin{key} Paired -- artificially by matching case to control.
\end{key}
  

\item Each student in an intro stat class walked 100 yards twice. Once
  with arms down at his/her sides, another time while ``pumping'' arms
  up and down with each stride. The order of ``pumping'' or ``not''
  was randomized for each student.  Pulse (beats per second) was
  measured after each walk.
\begin{students}
 \vspace{1cm}
\end{students}

\begin{key} Paired. Two measurements on each student.
\end{key}


    \end{enumerate}
  \end{enumerate}
  {\bf Being Careful With Wording}:\\
  With paired data, we are making inference about the ``true mean of
  the differences.''  This is different from the wording we used with
  two independent samples. There we looked at the ``difference in true
  means'' -- which makes sense because we had two populations to
  compare. With paired data, we have a single sample of
  differences. The observations subtracted to get the differences were
  not independent because they came from the same subject, but the
  sample of all differences can be independent as long as one subject
  (or unit) did not influence another. 



   Now we'll analyze some paired data.

   \begin{center}
     {\large\bf Tears and Testosterone}
   \end{center}
    Do pheromones (subconscious chemical signals) in female
      tears {\bf change}  testosterone levels in men?\\
    Cotton pads had either real female tears or a salt solution
      that had been dripped down the same woman's face.
    Fifty men (a convenience sample) had a pad attached to their upper
    lip twice, once 
      with tears and once without, in random order.
    Response variable: testosterone level measured in
      picograms/milliliter, pg/ml.

    Take differences: Saline T--level minus Tears T--level.\\
    The mean of the differences is  $\overline{x}_D = -21.7$, and
    the spread of the differences is  $s_D = 46.5$ pg/ml.
    \begin{enumerate}
  \setcounter{enumi}{3}
        \item Test: ``Is the mean difference 0?''
          \begin{enumerate}
            \item Check the assumptions. If the differences are
              fairly symmetrically distributed, can we use t-procedures?
              Explain. 
\begin{students}
    \vspace{2.8cm}    
\end{students}

\begin{key}
  {\it  Yes. $n=50 > 30$ so we're OK. We have random ordering, and 
     subjects' responses should be independent.} 
\end{key}

        \item State hypotheses in terms of parameter $\mu_D$, the
              true mean of differences.
\begin{students}
    \vspace{2cm}    
\end{students}

\begin{key}
  {\it  $H_0: \mu_D = 0$ versus $H_a: \mu_D \neq 0$}
\end{key}

    {\bf STOP.  Check} the direction of the alternative with another group.
    \item Compute t statistic as for  testing $H_0:\ \mu_D=0$.
\begin{students}
    \vspace{1.5cm}    
\end{students}

\begin{key}
  {\it  $t^* = \frac{ -21.7}{46.5/\sqrt{50}} = -3.30 $}
\end{key}

            \item Do we use Normal  or t distribution? 
              If t how many df?
\begin{students}
    \vspace{1cm}    
\end{students}

\begin{key}
  {\it     $t$ with 49 df}
\end{key}

            \item Look up the p-value in a web app and give the
              strength of evidence.
\begin{students}
    \vspace{1.7cm}    
\end{students}

\begin{key}
  {\it  $2 \times .0009 = 0.0018$ This is very strong evidence against
  the null hypothesis.}
\end{key}

   \item At the $\alpha = .02$ level, what is your decision?
\begin{students}
    \vspace{1cm}    
\end{students}

\begin{key}
  {\it  Reject $H_0$.}
\end{key}


            \item Explain what we've learned in context. (Give scope
              of inference.  If outside observers would say the men
              studied were representative of all US men, how could the
              scope be extended?)
\begin{students}
    \vspace{1cm}    
\end{students}

\begin{key}
  {\it  We've found causal evidence that exposure to womens' tears
    does reduce testosterone levels in men. (p-value = 0.002). If this
  is a representative sample, then we can extend the inference back to
the population of US adult males.}
\end{key}

          \end{enumerate}
        \item Build a 90\% CI for the true mean difference, $\mu_D$.
          \begin{enumerate}
          \item Find the correct multiplier.
\begin{students}
    \vspace{1cm}    
\end{students}

\begin{key}
  {\it     $t^*_{49}  = 1.677 $ }
\end{key}
   \item  Compute the margin of error and build the interval.
\begin{students}
    \vspace{2cm}    
\end{students}

\begin{key}
$$ME = 1.677\times 46.5/\sqrt{50} = 1.677\times 6.58 = 11.03;
    \mbox{\hspace{.4in} CI: } -21.7 \pm 11.03 = (-32.7, -10.7) pg/ml$$
\end{key}

 \item Interpret the interval in context. What do you mean by ``confidence''?
\begin{students}
    \vspace{5cm}    
\end{students}

\begin{key}
  {\it  We are 90\% confident that the true mean difference in
    testosterone levels for men without and with exposure to womens'
    tears is in the interval (-32.7, -10.7) pg/ml. This says that the
    tears really did cause a decrease in T--levels in this group of
    men. Our confidence is in the process by which we built the
    interval.  When the procedure is used over and over, 90\% of
    intervals built this way (in the long run) will include their true
    mean.  }
\end{key}
          \end{enumerate}

   \item Designing a study: Researchers at the Western Transportation
     Institute (just south of the football stadium) use a driving
     simulator to test driver distractions (among other things). It is
     the front end of a car with large projection screens.  Suppose 
     they want to assess how distracting it is to read text messages
     on a phone while driving. The response they measure will be
     response time when a child suddenly runs out onto the road in
     front of the car.
     \begin{enumerate}
      \item How would they set this up to use  paired measurements?
\begin{students}
    \vspace{1cm}    
\end{students}

\begin{key}
  {\it Each driver does the same course twice -- once while reading a
    text and again without the phone.}
\end{key}
\item How could randomization be used?
\begin{students}
    \vspace{1cm}    
\end{students}

\begin{key}
  {\it Randomly (flip a coin?) select whether each driver gets the
    text reading first or second.}
\end{key}
\item Alternately, they could just test half their subjects
        while reading a text and half without the texting.  Which
        study design do you recommend?  Explain why.
\begin{students}
\vspace{4cm}
\end{students}

\begin{key}
  {\it Pairing seems like a good idea because some people have faster
    reaction times than others. By subtracting two reaction times you
    get rid of other effects such as age or sleepiness. }
\end{key}
\end{enumerate}

\item A study of the effects of drinking diet soda versus
  sugar--sweetened soda on weights of 32 adults used a ``cross--over''
  design in which each adult was given diet soda to drink for one six
  week period, and given sugar--sweetened soda to drink in another six
  week period.  The order of assignment was randomized for each
  participant.  The response measurement was ``weight gain'' over the
  six weeks.
  \begin{enumerate}
  \item It the parameter of interest the difference in true means or
    the true mean difference?  Explain.
\begin{students}
\vspace{3cm}
\end{students}

\begin{key}
  {\it True mean difference. This is a case of one sample measured
    twice.  We subtract weight gains (say sugar minus diet) so that
    our observations are independent of each other.}
\end{key}
\item How would you process the data in order to analyze it? 
\begin{students}
\vspace{2cm}
\end{students}

\begin{key}
  {\it Take difference in weight gain (diet minus sugar) for each person.}
\end{key}
\item What distribution would you use to find p-values or
    a confidence interval multiplier?
\begin{students}
\vspace{2cm}\newpage
\end{students}

\begin{key}
  {\it $t_{31}$}
\end{key}
\end{enumerate}

\end{enumerate}


\begin{center}
  {\large\bf Take Home Message}
\end{center}
\begin{itemize}
\item Taking two measurements on the same subjects is quite different
  from taking two samples or assigning two treatments. You need to
  read carefully to see how data were collected.  We do not have
  independent measures if they are on the same units.

  \item With two independent samples the parameter of interest was
    ``difference in true means.''  With a single sample measured
    twice, the parameter is ``true mean difference''.

  \item To analyze paired measures, take differences first (before
    averaging) and use the one-sample t procedures.

  \item Pairing is a good strategy for reducing variability.

  \item The Energy Drinks study has a lot going on.  They took
    differences first, to get ``change in RBANS'', but they also had
    three independent groups to compare: Control, RED, and REDA.

  \item What  questions do you have?  Write them here.\vfill
\end{itemize}


%% need more -- only 2 pages.

\begin{center}
  {\large\bf Assignment}
\end{center}

\begin{itemize}
\item Review for the final exam.
%\item D2Box 11 (last one) is due April 21st Turn in a pdf on D2L.
%\item D2Quiz 12 (last one) is due April 25.  Fill in your answers on D2L.
 %%  We strongly encourage you to get help in the Math Learning Center.
\item Read the next three pages before your next class.
\item Check D2L for videos and other assignemnts.
\end{itemize}
