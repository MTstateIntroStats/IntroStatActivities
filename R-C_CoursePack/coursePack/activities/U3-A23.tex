\def\theTopic{Difference in Proportions - Z }
\def\dayNum{23 }


\section{Normal Inference - Difference in Proportions}



When we did a hypothesis test to see if the difference in  two true
proportions was zero,  for example when evaluating the effectiveness
of peanut protein, we shuffled cards and relabeled the two groups many
times. Now we'll use the normal distribution instead.  
  

 To make the switch from simulations to a theoretical distribution,
 we need, just as for a single proportion, a formula for the standard
 error of our statistic.  In the last activity our statistic was
 $\widehat{p}$ and our formula was $SE(\widehat{p}) =
 \sqrt{\widehat{p}(1-\widehat{p})/n}$.  To compare two groups, our 
statistic is $\widehat{p}_1 - \widehat{p}_2$ and we need a formula for
$SE(\widehat{p}_1 - \widehat{p}_2)$. As with a single proportion, the
form of this standard error depends on whether we are doing a
hypothesis test (assuming some $H_0$ is true) or building a confidence
interval.  We'll start with the hypothesis test which is typically
testing to see if the two groups have the same true proportion, that
is:  $H_0:\ p_1 = p_2$.
 \begin{itemize}
 \item If $H_0$ is true, the two groups are really the same, and we
   should combine them to get a better estimate of the overall
   proportion of successes. We'll call estimate $\widehat{p}_T$ where
   the ``T'' stands for ``{\bf total}'' or ``{\bf marginal}'' because it's
   based on totals which appear in the outer margin of a table.  We
   find it by combining all successes from both 
   groups, then dividing by the sum of the sample sizes. 
  $$\widehat{p}_T = \frac{x_1 + x_2}{n_1+n_2} = \frac{n_1\widehat{p}_1
    + n_2\widehat{p}_2}{n_1 + n_2}$$ 
  Recall:  we used the same combined estimate when simulating draws
  from $H_0$ earlier in Activity 12.
 \end{itemize}

 The {\bf hypothesis testing} formula for standard error of the
 difference in sample proportions is: 
 $$SE(\widehat{p}_1 - \widehat{p}_2) =
 \sqrt{\frac{\widehat{p}_T(1-\widehat{p}_T)}{n_1} +
   \frac{\widehat{p}_T(1-\widehat{p}_T)}{n_2}} =
  \sqrt{\widehat{p}_T(1-\widehat{p}_T)\left(\frac{1}{n_1} +
   \frac{1}{n_2}\right)}$$


\begin{center}
\vspace*{.1in}
{\bf {\large Are we better people in a pleasant environment?}}
\end{center}
\vspace{-.1in}

For class today, you read an article abstract about a study involving
the smell of baking bread.  Answer these questions about it:


 \begin{enumerate}
   \item  Was a random mechanism used to select the person studied?
     Explain. 
\begin{students}
 \vspace{1cm}\\
\end{students}
\begin{key}
\\ {\it Yes, if the coin flip was performed, it would randomly select
  about half of the people walking by. }
\end{key}
   \item  What was the “treatment” and how was it applied to a
     subject? 
\begin{students}
 \vspace{1cm}\\
\end{students}
\begin{key}
 \\{\it The ``treatment'' was the location: either near a bakery or
   near another type of store. It was ``set'', but not for each
   subject. Choice of treatment filtered out the potential subjects.
   It was not applied at random.}
\end{key}
   \item  Does this study fit the definition of an experiment, or is
     it observational? Explain. 
\begin{students}
 \vspace{1cm}\\
\end{students}
\begin{key}
 \\ {\it I'd say it's observational, since a given passerby probably
   was a possible subject for just one of the treatments, not both.}
\end{key}
   \item  Name three or more possible lurking variables. 
\begin{students}
 \vspace{1cm}\\
\end{students}
\begin{key}
 \\ {\it Reason for visiting the mall (need bread? or need
   clothing?).  Gender. Socio-economic status. Tendency to lose things.}

\end{key}
   \item  What is the scope of inference for this study? 
\begin{students}
 \vspace{1cm}\\
\end{students}
\begin{key}
 \\ {\it  We can only infer association within this sample because the
   subjects where only haphazardly selected, and treatments were not
   randomly applied.}
\end{key}

\item \label{Bake-hypotheses}What are null and alternative hypotheses
  for this study? Assume that researchers were not willing to state
  ahead of time whether a good smell makes people ``better'' or ``worse''. \\
  $H_0:$\
\begin{students}
 \vspace{1cm}\\
\end{students}
\begin{key}
 $p_1 = p_2$ \\
\end{key}
   $H_a: $
\begin{students}
     \vspace{1cm}\\
\end{students}
\begin{key}
   $ p_1 \neq p_2$ \\
\end{key}   

     Check you  answers just above with other groups at your table.
     Do we all agree about the direction of the alternative?  
\item Compute the following proportions:\\
    Bakery group:  $\widehat{p}_1 = $
\begin{students}
 \vspace{1cm}\\
\end{students}
\begin{key}
  $154/200 = 0.752$ \\
\end{key}
Clothing store: $\widehat{p}_2 = $
\begin{students}
 \vspace{1cm}\\
\end{students}
\begin{key}
  $104/200 = 0.52$ \\
\end{key}
    Overall: $\widehat{p}_T = $
\begin{students}
 \vspace{1cm}\\
\end{students}
\begin{key}
  $258/400 = 0.645$ \\
\end{key}

\item When testing one proportion, we created a $z$ statistic with
$z = \frac{\widehat{p} - p_0}{SE(\widehat{p})}$.  In general, we use
$$ z = \frac{\mbox{statistic - null value}}{SE(\mbox{statistic})}$$
Now our statistic is $\widehat{p}_1 - \widehat{p}_2$.  
\begin{itemize}
\item What value do we expect it to have if $H_0$ is true?
\begin{students}
 \vspace{1cm}\\
\end{students}
\begin{key}
  $0$ \\
\end{key}
\item What is the standard error of the statistic under $H_0$?
\begin{students}
 \vspace{1cm}\\
\end{students}

\begin{key}
  $SE(\widehat{p}_1 - \widehat{p}_2 ) = \sqrt{ 0.645\times 0.355
    (\frac{1}{200} +  \frac{1}{200})} = \sqrt{.002281} = 0.045$ 
\end{key}

\item Compute our $z$ statistic.
\begin{students}
 \vspace{1cm}\\
\end{students}
\begin{key}
 $z = \frac{ 0.752 - 0.52}{0.04776} = \frac{0.25}{0.045} = 5.1$ \\
\end{key}  
\end{itemize}
\item Use the web app to find the probability.  
How strong is the evidence against $H_0$?
\begin{students}
 \vspace{1cm}\\
\end{students}
\begin{key}
  \\ {\it p-value $ \leq 2 \times(0.00001) = 0.00002$ This is very, very
    strong evidence refuting the null hypothesis that people act just
    as helpfully in the two situations.  In fact, the people close to
    the bakery were much more helpful than those by the clothing
    store.  }
\end{key}

\item Write up the results as a statistical report on your own paper.
  (Suggestion:  finish the activity, then come back to this.) 
\begin{students}
 \vspace{1cm}
\end{students}

  \end{enumerate}

\subsection{ Confidence Interval for the Difference in True Proportions}

Next we want to get an interval estimate of the difference in true
proportions.  We'll use the same data to ask: ``How much more helpful
are people near a bakery than near a clothing store?''

Again, looking back to a single proportion we used $\widehat{p} \pm
z^*SE(\widehat{p})$ which is a special case of the general rule:
$$ \mbox{estimate} \pm \mbox{multiplier} \times SE(\mbox{estimate})$$

All we need to do is to find the $SE(\widehat{p}_1 - \widehat{p}_2)$.
We {\bf do not assume the two are equal}, so no $\widehat{p}_T$ is
needed. The formula is:
  $$ SE(\widehat{p}_1 - \widehat{p}_2) = \sqrt{ 
      \frac{\widehat{p}_1(1 - \widehat{p}_1)}{n_1} + 
      \frac{\widehat{p}_2(1 - \widehat{p}_2)}{n_2}} $$
You might ask why there is a plus sign between the two terms inside
the square root, but a minus sign in the estimator.  It's because each
sample  proportion has some variability, and $\widehat{p}_1 - \widehat{p}_2 $ 
can vary due to changes in $\widehat{p}_1$ or in $\widehat{p}_2$.  Subtracting
would imply that having a second sample  makes the difference {\bf
  less} variable, when really it makes our statistic {\bf more}
variable. 

OK, we are now ready to build a confidence interval.

\begin{enumerate}
  \setcounter{enumi}{11}
  \item Use $ \widehat{p}_1$ and $\widehat{p}_2$ to compute the standard
    error of the difference in sample proportions.
\begin{students}
  \vspace{2cm}\\
\end{students}
\begin{key}
$ SE(\widehat{p}_1 - \widehat{p}_2) = \sqrt{ 0.752\times(1- 0.752)/200
  + 0.502\times(1-  0.502)/200} =  0.046 $
\end{key}

\item In this case, a 90\% confidence interval is needed.  Refer back
  to your table from last class, or use the web app
 % (\url{http://shiny.math.montana.edu/prob}) 
  to find $z^*$.  Find the margin of error and   build the interval.
\begin{students}
 \vspace{1cm}\\
\end{students}
\begin{key}
ME = $ 1.645 \times 0.046 = 0.076 $ 90\% CI =  $0.752 - 0.502 \pm
0.076 = 0.25 \pm 0.076 = (0.174, 0.326)$
\end{key}

\item Interpret the CI in the context of this research question.
\begin{students}
 \vspace{1cm}\\
\vspace{1in}
\end{students}
\begin{key}
 {\it We are 90\% confident that the true proportion of helpful people
   near a bakery is 17.4 to 32.6\% higher than near a clothing store.}
\end{key}
\end{enumerate}

\begin{center}
  {\large\bf Assumptions?}
\end{center}

  We need basically the same assumptions when working with two samples
  as with one sample proportion.  The first two apply to any method
  of doing hypothesis tests or confidence intervals.  The last is
  particular for normal-theory based methods with proportions. 
  \begin{itemize}
     % \item The size of each sample must be less than a tenth the size
     %   of its population.
     \item Each sample must be representative of its population. 
     \item {\bf Independent} responses.  Definition: responses are
       independent if knowing one response does not help us guess the
       value of the other.  Sampling multiple people from the same
       household gives {\bf dependent} responses.  If we have a random
       sample, we can assume observations are independent.
     \item To use normality for a confidence interval: at least 5
       successes and 5 failures in each group.  To use normality for
       hypothesis testing, take the smaller of $n_1$ and $n_2$ times
       the smaller of $\widehat{p}_T$ or ($1-\widehat{p}_T$) and this
       value should be at least 5.\vspace{1in}
     \item Groups must be independent -- no units are included in both
       groups. 
  \end{itemize}



\begin{center}
  {\large\bf Take Home Messages}
\end{center}

\begin{itemize}
 \item  To do hypothesis testing we needed the ``overall'' estimated
   success proportion -- forgetting about the two groups. 
 \item Our estimate of spread, the $SE$, changes depending on whether
   we assumed $p_1=p_2$, as in hypothesis testing, or not (confidence
   intervals). Know both versions.
 \item  The general form of a standardized statistic for hypothesis
   testing is:
     $$z = \frac{\mbox{statistic - null value}}{SE(\mbox{statistic})}$$
in this case that is
$$ z = \frac{\widehat{p}_1 - \widehat{p}_2 -0}{ SE(\widehat{p}_1 -
  \widehat{p}_2)} = \frac{\widehat{p}_1 - \widehat{p}_2} 
   {\sqrt{\frac{\widehat{p}_T(1-\widehat{p}_T)}{n_1} +
    \frac{\widehat{p}_T(1-\widehat{p}_T)}{n_2}}}$$
\item To build a confidence interval, we do not assume ${p}_1 ={p}_2$.
\item The general form of a CI is
$$ \mbox{estimate} \pm \mbox{multiplier} \times SE(\mbox{estimate})$$
    which in this case is
  $$ \widehat{p}_1 - \widehat{p}_2 \pm z^* SE(\widehat{p}_1 -
  \widehat{p}_2) = \widehat{p}_1 - \widehat{p}_2 \pm z^* 
\sqrt{\frac{\widehat{p}_1(1 - \widehat{p}_1)}{n_1} + 
      \frac{\widehat{p}_2(1 - \widehat{p}_2)}{n_2}}$$
\end{itemize}\vfill




\begin{center}
  {\large\bf Assignment}
\end{center}

\begin{itemize}
\item Fill in the bottom 3 boxes in column 3 of the Review Table. 
\item D2Box 10 is due Nov 17.  
\item D2Quiz 11 is due Nov 28.  
%% Turn it in as a pdf file to the   DropBox on D2L.
 %%  We strongly encourage you to get help in the Math Learning Center.
\item Watch videos assigned on D2L.
%\item Watch  videos 12 and 13 under Unit 3 Examples.
\item Read the next two pages before your next class.
\end{itemize}
