\def\theTopic{Reading 20 }

\section{ Combining Lots of ``Small'' Choices}

Watch this video:\\
\url{https://www.youtube.com/watch?v=6YDHBFVIvIs}


Questions:
\begin{itemize}
\item What happens (and with what probability) when a ball hits a
  nail?  \vspace{2cm}
\item Is the path of one ball predictable?  \vspace{2cm}
\item What does the narrator say ``is predictable''?  \vspace{2cm}
\item Why do more balls end up in the middle than at the edges?   \vspace{2cm}
\item What is the pattern we get after dropping many balls?  \vspace{2cm}
\item What examples from nature follow that pattern?  \vspace{2cm}
\end{itemize}

The process of averaging numbers together is similar to watching one
ball drop through Galton's board.

Instead of nails pushing the ball left or right, each nail represents
one unit in the sample.   If the individual is ``large'' (in whatever
scale we are measuring), the ball is pushed to the right.  If the unit
is ``small'', then the sample mean gets a little smaller as well --
moving to the left. The amount the ball moves is not exactly one unit,
like in the demo, because a really big or really small unit could pull
quite a bit farther. 

\begin{itemize}
\item What type of units would force the ``averaging ball'' to end up
  at the far left?  or far right?  \vspace{2cm}
\item Are there still more paths to the middle than to the extremes?
  \vspace{2cm} 
\end{itemize}

Watch the ``Wisdom of the crowd'' video:\\
\url{https://www.youtube.com/watch?v=uz5AeHqUtRs }

\begin{itemize}
\item The Galton's Board video talked about two things:
  \begin{itemize}
  \item         that balls tend to end up in the middle and
  \item         that we get a particular pattern from dropping many balls.
  \end{itemize}
The second video illustrates only one of those points.  Which one? \vspace{1cm}

\item How could we find the pattern of crowd guesses?\vspace{2cm}
\end{itemize}


